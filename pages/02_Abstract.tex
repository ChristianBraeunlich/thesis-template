%%%%%%%%%%%%%%%%%%%%%%%%%%%%%%%%%%%%%%%%%%%%%%%%%%%%%%%%%%%%%%%%%%%%%%%%%%%%
%%%%%%%%%%%%%%%%%%                Abstract                   %%%%%%%%%%%%%%%
%%%%%%%%%%%%%%%%%%%%%%%%%%%%%%%%%%%%%%%%%%%%%%%%%%%%%%%%%%%%%%%%%%%%%%%%%%%%
\chapter*{Abstract}
%\addcontentsline{toc}{chapter}{Abstract}
\thispagestyle{empty}
Text-Mining: Auswertung unstrukturierter (textueller) Informationen.

Pendent dazu wäre das Data-Mining: Auswertung strukturierter Informationen.

Aus den unterschiedlichen Untersuchungsgegenständen, ergeben sich jedoch andere Schwerpunkte bei der Vorverarbeitung:

- Beim Data-Mining steht die Bereinigung, Normalisierung der Daten und der Verbindung unterschiedlichster Daten im MiOelpunkt.

- Beim Text-Mining steht die Erkennung und ExtrakKon repräsentaKver Konstrukte im MiOelpunkt

Disziplinen

Clustering, Klassifikation, Mustererkennung, Information Retrieval, Domänenwissen

Folie 01-15

Hauptaufgaben des Text-Mining Prozesses nach Zhang und Segall (2008)

oder 

Hauptaufgaben des Text-Mining Prozesses nach Hotho et al. (2005)

\begin{itemize}
    \item Ein Problem identifizieren und analysieren.
    \item Verschiedene Lösungsansätze identifizieren, gegenüberstellen und bezüglich des Problems bewerten.
    \item Den gewählten Lösungsansatz umsetzen und dokumentieren.
\end{itemize}

