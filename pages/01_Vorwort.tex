\chapter*{Vorwort}
%\addcontentsline{toc}{chapter}{Vorwort}
\thispagestyle{empty}

\begin{displayquote}
\textit{Niemand meint alles was er sagt.}
\end{displayquote}

\textbf{Was ist Merkmalsextraktion?}
Merkmalsextraktion ist ein Prozess der Dimensionalitätsreduktion, bei dem ein anfänglicher Satz von Rohdaten zur Verarbeitung auf besser handhabbare Gruppen reduziert wird. Ein Merkmal dieser großen Datensätze ist eine große Anzahl von Variablen, deren Verarbeitung eine Menge an Rechenressourcen erfordert. Merkmalsextraktion ist die Bezeichnung für Methoden, die Variablen auswählen und/oder zu Merkmalen kombinieren, wodurch die Menge der zu verarbeitenden Daten effektiv reduziert wird, während der ursprüngliche Datensatz immer noch genau und vollständig beschrieben wird.

\noindent
\textbf{Warum ist dies nützlich?}

Der Prozess der Merkmalsextraktion ist nützlich, wenn Sie die Anzahl der für die Verarbeitung benötigten Ressourcen reduzieren müssen, ohne wichtige oder relevante Informationen zu verlieren. Die Merkmalsextraktion kann auch die Menge der redundanten Daten für eine bestimmte Analyse reduzieren. Außerdem erleichtern die Reduzierung der Daten und die Bemühungen des Rechners, variable Kombinationen (Merkmale) zu bilden, die Geschwindigkeit der Lern- und Generalisierungsschritte im maschinellen Lernprozess.

\textbf{Praktische Anwendungen der Merkmalsextraktion}

\textbf{Autoencoder}
- Der Zweck von Autoencodierern ist das unbeaufsichtigte Erlernen einer effizienten Datencodierung. Die Merkmalsextraktion wird hier verwendet, um Schlüsselmerkmale in den zu kodierenden Daten zu identifizieren, indem aus der Kodierung des ursprünglichen Datensatzes gelernt wird, um neue Merkmale abzuleiten.

\textbf{Worttasche}
- Eine Technik zur Verarbeitung natürlicher Sprache, die die in einem Satz, Dokument, einer Website usw. verwendeten Wörter (Merkmale) extrahiert und nach der Häufigkeit ihrer Verwendung klassifiziert. Diese Technik kann auch in der Bildverarbeitung angewendet werden.

Im Anschluss dieser Bachelorarbeit wird ein Fazit aus den Erkenntnissen gezogen.

