\chapterimage{media/ChapterImage.png}
\chapter{Identifizierung von Anforderungen}
\label{cha:durchfuehrungfeature}

\epigraph{Whether you are analyzing social media, feeds, HTML and PDF documents, or any other kinds of text, the first step involves converting the content to raw text.}{VISHAL DESHPANDE}

\lipsum[3]

\section{Datenbeschaffung}
\label{sec:datenbeschaffung}

\lipsum[3]

\begin{savenotes}
    \begin{table}[!ht]
        \small
        \centering
        \begin{tabular}{llllll}
        \hline
        \#                      & Titel         & \# Anforderungen         & Beschaffenheit   & \# Seiten     \\ \hline
        1                       & PH\_A01       & 175 & 1                & 18            \\
        \rowcolor[HTML]{EFEFEF} 
        2                       & PH\_A02 - Brettspiel       & 88 & 2                & 24            \\
        3                       & PH\_A03 - Sportportal       & 255 & 3                & 51            \\
        \rowcolor[HTML]{EFEFEF}
        4                       & PH\_A04 - Handyverträge     & 32   & 3                & 23            \\
        5                       & PH\_KTC01     & 131 & 3                & 34            \\
        \rowcolor[HTML]{EFEFEF}
        6                       & PH\_KTC02     & 33 & 2                & 16            \\   
        \end{tabular}
        \caption{Überblick über die Beschaffenheit der Pflichtenhefte}
        \label{tab:pflichtenhefte}
    \end{table}
\end{savenotes}

\section{Text-Extrahierung}
\label{sec:text-extrahierung}

\lipsum[3]

\section{Text-Vorverarbeitung}
\label{sec:text-vorverarbeitung}

\lipsum[3]

\section{Anwendung des SAFE-Ansatzes}
\label{sec:pos-muster-analyse}

\lipsum[3]

\section{Optimierung des SAFE-Ansatzes}
\label{optimierung}

\lipsum[3]

\begin{table}[!ht]
    \centering
    \begin{adjustbox}{width=0.95\columnwidth, center}
    \resizebox{\textwidth}{!}{
        \begin{tabular}{lll}
        \hline
        \textbf{\#} & \textbf{POS Muster}                                 & \textbf{POS-Tags Muster}\\ \hline
        1  & Determinante Nomen Verb                    & DET-NOUN-VERB\\
        \rowcolor[HTML]{EFEFEF} 
        2  & Pronomen Verb - Adjektiv                   & PRON-VERB-{}-ADV\\
        3  & Nomen Verb Adjektiv Verb                   & NOUN-VERB-ADJ-VERB\\
        \rowcolor[HTML]{EFEFEF}
        4  & Nomen Verb Adposition Nomen Verb           & NOUN-VERB-ADP-NOUN-VERB\\
        5  & Nomen Adposition Determinante Nomen Verb   & NOUN-ADP-DET-NOUN-VERB\\
        \rowcolor[HTML]{EFEFEF}
        6  & Nomen Adposition Determinante Nomen Verb   & NOUN-ADP-DET-NOUN-VERB\\
        7  & Adposition Determinante Nomen Verb         & ADP-DET-NOUN-VERB\\
        \rowcolor[HTML]{EFEFEF}
        8  & Adjektiv Verb Determinante Nomen Determinante & ADJ-VERB-DET-NOUN-DET\\
        9  & Determinante Nomen Hilfsverb               & DET-NOUN-AUX\\
        \rowcolor[HTML]{EFEFEF}
        10 & Determinante Nomen Verb Nomen              & DET-NOUN-VERB-NOUN\\
        11 & Determinante Nomen Verb Adposition         & DET-NOUN-VERB-ADP\\
        \rowcolor[HTML]{EFEFEF}
        12 & Determinante Nomen Verb Determinante       & DET-NOUN-VERB-DET\\
        \end{tabular}
    }
    \end{adjustbox}
    \caption{Überblick über die eigenen POS-Muster}
    \label{tab:pos}
\end{table}

\lipsum[2]
