\chapterimage{media/ChapterBackground.png}
\chapter{Ergebnis}

\lipsum[3-5]

\chapter{Test}

\lipsum[3-5]

\section{Agile Software Development}

\lipsum[3-5]

\subsection{Erste Verwendung}

\lipsum[3-5]

\subsubsection{EPICs}

The \Gls{nlp} typesetting markup language is specially suitable 
for documents that include \gls{text-mining}. \Glspl{feature_extraction} are 
rendered properly an easily once one gets used to the commands.

Given a set of numbers, there are elementary methods to compute 
its \acrlong{gcd}, which is abbreviated \acrshort{gcd}. This 
process is similar to that used for the \acrfull{lcm}.

\subsubsection{Featues}

\lipsum[3-5]

\section{YouTrack}

Attribute: \footnote{Dies ist ein Beispiel}

\lipsum[3-5]

