%%%%%%%%%%%%%%%%%%%%%%%%%%%%%%%%%%%%%%%%%%%%%%%%%%%%%%%%%%%%
%%%%%%%%%%%              Packages                %%%%%%%%%%%
%%%%%%%%%%%%%%%%%%%%%%%%%%%%%%%%%%%%%%%%%%%%%%%%%%%%%%%%%%%%
%
% ______________ Packages __________________________________
%   > Definition of global Parameters.
%

% ____User-Informations_____________________
% 		- Definition of global Parameters.
%
% ==========================================

% ____ Umlaute _____________________________
% 	Umlaute/Sonderzeichen wie äüöß direkt im Quelltext verwenden (CodePage).
%		Erlaubt automatische Trennung von Worten mit Umlauten.
% ==========================================
\usepackage{microtype} % Slightly tweak font spacing for aesthetics
\usepackage[utf8]{inputenc}
\usepackage[T1]{fontenc}

% ____ Schriftart __________________________
%   > Arial
% ==========================================
\usepackage{helvet}
\renewcommand{\familydefault}{\sfdefault}

\usepackage{lmodern}
%\usepackage{ae} % "schöneres" ß
\usepackage{textcomp} % Euro-Zeichen etc.

% Anpassung an Landessprache -----------------------------------------------
% 	Verwendet globale Option german siehe \documentclass
% --------------------------------------------------------------------------
%
% ______________ Packages ____________________________________
%   > Definition of global Parameters.
%
\usepackage{babel}


\usepackage{emptypage}


\usepackage{tocbasic}

% Quotation Package
\usepackage{csquotes}

% Bessere Unterstreichungen ---------------------------------------------
\usepackage[normalem]{ulem}


% Abkürzungsverzeichnis
\usepackage[acronym, toc]{glossaries}
\makeglossaries

% Grafiken -----------------------------------------------------------------
% 	Einbinden von EPS-Grafiken [draft oder final]
% 	Option [draft] bindet Bilder nicht ein - auch globale Option
% --------------------------------------------------------------------------
\usepackage[dvips, final]{graphicx}
\graphicspath{{media/}} % Dort liegen die Bilder des Dokuments

% Befehle aus AMSTeX für mathematische Symbole z.B. \boldsymbol \mathbb ----
\usepackage{amsmath,amsfonts}

%
% Zeilenumbruch bei Bildbeschreibungen
%
\setcapindent{1em}

% Für Index-Ausgabe; \printindex -------------------------------------------
\usepackage{makeidx}

% Symbolverzeichnis --------------------------------------------------------
% 	Symbolverzeichnisse bequem erstellen, beruht auf MakeIndex.
% 		makeindex.exe %Name%.nlo -s nomencl.ist -o %Name%.nls
% 	erzeugt dann das Verzeichnis. Dieser Befehl kann z.B. im TeXnicCenter
%		als Postprozessor eingetragen werden, damit er nicht ständig manuell
%		ausgeführt werden muss.
%		Die Definitionen sind ausgegliedert in die Datei Abkuerzungen.tex.
% --------------------------------------------------------------------------
\usepackage[intoc]{nomencl}
  \let\abbrev\nomenclature
  \renewcommand{\nomname}{Abkürzungsverzeichnis}
  \setlength{\nomlabelwidth}{.25\hsize}
  \renewcommand{\nomlabel}[1]{#1 \dotfill}
  \setlength{\nomitemsep}{-\parsep}

% zum Umfließen von Bildern ---------------------------------------------------------
\usepackage[vflt]{floatflt}
\usepackage{subfigure}

% Zum Einbinden von Programmcode --------------------------------------------
\usepackage{listings}
\usepackage{xcolor} 
\definecolor{hellgelb}{rgb}{1,1,0.9}
\definecolor{colKeys}{rgb}{0,0,1}
\definecolor{colIdentifier}{rgb}{0,0,0}
\definecolor{colComments}{rgb}{1,0,0}
\definecolor{colString}{rgb}{0,0.5,0}
\lstset{%
    float=hbp,%
    basicstyle=\texttt\small, %
    identifierstyle=\color{colIdentifier}, %
    keywordstyle=\color{colKeys}, %
    stringstyle=\color{colString}, %
    commentstyle=\color{colComments}, %
    columns=flexible, %
    tabsize=2, %
    frame=single, %
    extendedchars=true, %
    showspaces=false, %
    showstringspaces=false, %
    numbers=left, %
    numberstyle=\tiny, %
    breaklines=true, %
    backgroundcolor=\color{hellgelb}, %
    breakautoindent=true, %
%    captionpos=b%
}

% Lange URLs umbrechen etc. -------------------------------------------------
\usepackage{url}

% Paket zum sauberen Einbauen von externen PDF-Dateien -----------------
\usepackage[final]{pdfpages}

% Zum fortlaufenden Durchnummerieren der Fußnoten ---------------------------
\usepackage{chngcntr}

% Beschriftung von Tabellen und Bildern ändern ----------------------------------------------------------
\addto\captionsngerman{
	\renewcommand{\figurename}{Abb.}
	\renewcommand{\tablename}{Tab.}
}

% Rotation von Elementen -------------------------------------------------------
\usepackage{rotating}

% für lange Tabellen
\usepackage{longtable}
\usepackage{array}
\usepackage{ragged2e}
\usepackage{lscape}

%Spaltendefinition rechtsbündig mit definierter Breite
\newcolumntype{z}[1]{>{\raggedleft\hspace{0pt}}p{#1}}

% Formatierung von Listen ändern
\usepackage{paralist}
% \setdefaultleftmargin{2.5em}{2.2em}{1.87em}{1.7em}{1em}{1em}

% Anhangsverzeichnis
%\makeatletter% --> De-TeX-FAQ
%\newcommand*{\maintoc}{% Hauptinhaltsverzeichnis
%\begingroup
%\@fileswfalse% kein neues Verzeichnis öffnen
%\renewcommand*{\appendixattoc}{% Trennanweisung im Inhaltsverzeichnis
%\value{tocdepth}=-10000 % lokal tocdepth auf sehr kleinen Wert setzen
%}%
%\tableofcontents% Verzeichnis ausgeben
%\endgroup
%}
%\newcommand*{\appendixtoc}{% Anhangsinhaltsverzeichnis
%\begingroup
%\edef\@alltocdepth{\the\value{tocdepth}}% tocdepth merken
%\setcounter{tocdepth}{-10000}% Keine Verzeichniseinträge
%\renewcommand*{\contentsname}{% Verzeichnisname ändern
%Verzeichnis der Anh\"ange}%
%\renewcommand*{\appendixattoc}{% Trennanweisung im Inhaltsverzeichnis
%\setcounter{tocdepth}{\@alltocdepth}% tocdepth wiederherstellen
%}%
%\tableofcontents% Verzeichnis ausgeben
%\setcounter{tocdepth}{\@alltocdepth}% tocdepth wiederherstellen
%\endgroup
%}
%\newcommand*{\appendixattoc}
%\g@addto@macro\appendix{% \appendix erweitern
%\if@openright\cleardoublepage\else\clearpage\fi% Neue Seite
%\addcontentsline{toc}{chapter}{\appendixname}% Eintrag ins Hauptverzeichnis
%\addtocontents{toc}{\protect\appendixattoc}% Trennanweisung in die toc-Datei
%}
%\makeatother